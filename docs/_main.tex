%%%%%%%%%%%%%%%%%%%%%%%%%%%%%%%%%%%%%%%%%%%%%%%%%%%%%%%%%%%%%%%
%% OXFORD THESIS TEMPLATE

% Use this template to produce a standard thesis that meets the Oxford University requirements for DPhil submission
%
% Originally by Keith A. Gillow (gillow@maths.ox.ac.uk), 1997
% Modified by Sam Evans (sam@samuelevansresearch.org), 2007
% Modified by John McManigle (john@oxfordechoes.com), 2015
% Modified by Ulrik Lyngs (ulrik.lyngs@cs.ox.ac.uk), 2018-, for use with R Markdown
%
% Ulrik Lyngs, 25 Nov 2018: Following John McManigle, broad permissions are granted to use, modify, and distribute this software
% as specified in the MIT License included in this distribution's LICENSE file.
%
% John commented this file extensively, so read through to see how to use the various options.  Remember that in LaTeX,
% any line starting with a % is NOT executed.  Several places below, you have a choice of which line to use
% out of multiple options (eg draft vs final, for PDF vs for binding, etc.)  When you pick one, add a % to the beginning of
% the lines you don't want.


%%%%% PAGE LAYOUT
% The most common choices should be below.  You can also do other things, like replacing "a4paper" with "letterpaper", etc.

% This one formats for two-sided binding (ie left and right pages have mirror margins; blank pages inserted where needed):
%\documentclass[a4paper,twoside]{templates/ociamthesis}
% This one formats for one-sided binding (ie left margin > right margin; no extra blank pages):
%\documentclass[a4paper]{ociamthesis}
% This one formats for PDF output (ie equal margins, no extra blank pages):
%\documentclass[a4paper,nobind]{templates/ociamthesis}

% As you can see from the uncommented line below, oxforddown template uses the a4paper size, 
% and passes in the binding option from the YAML header in index.Rmd:
\documentclass[a4paper, nobind]{templates/curtinthesis}


%%%%% ADDING LATEX PACKAGES
% add hyperref package with options from YAML %
\usepackage[pdfpagelabels]{hyperref}
% handle long urls
\usepackage{xurl}
% change the default coloring of links to something sensible
\usepackage{xcolor}

\definecolor{mylinkcolor}{RGB}{0,0,139}
\definecolor{myurlcolor}{RGB}{0,0,139}
\definecolor{mycitecolor}{RGB}{0,33,71}

\hypersetup{
  hidelinks,
  colorlinks,
  linktocpage=true,
  linkcolor=mylinkcolor,
  urlcolor=myurlcolor,
  citecolor=mycitecolor
}



% add float package to allow manual control of figure positioning %
\usepackage{float}

% enable strikethrough
\usepackage[normalem]{ulem}

% use soul package for correction highlighting
\usepackage{color, soulutf8}
\definecolor{correctioncolor}{HTML}{CCCCFF}
\sethlcolor{correctioncolor}
\newcommand{\ctext}[3][RGB]{%
  \begingroup
  \definecolor{hlcolor}{#1}{#2}\sethlcolor{hlcolor}%
  \hl{#3}%
  \endgroup
}
\soulregister\ref7
\soulregister\cite7
\soulregister\citet7
\soulregister\autocite7
\soulregister\textcite7
\soulregister\pageref7

%%%%% FIXING / ADDING THINGS THAT'S SPECIAL TO R MARKDOWN'S USE OF LATEX TEMPLATES
% pandoc puts lists in 'tightlist' command when no space between bullet points in Rmd file,
% so we add this command to the template
\providecommand{\tightlist}{%
  \setlength{\itemsep}{0pt}\setlength{\parskip}{0pt}}
 
% UL 1 Dec 2018, fix to include code in shaded environments

% User-included things with header_includes or in_header will appear here
% kableExtra packages will appear here if you use library(kableExtra)
\usepackage{booktabs}
\usepackage{longtable}
\usepackage{array}
\usepackage{multirow}
\usepackage{wrapfig}
\usepackage{float}
\usepackage{colortbl}
\usepackage{pdflscape}
\usepackage{tabu}
\usepackage{threeparttable}
\usepackage{threeparttablex}
\usepackage[normalem]{ulem}
\usepackage{makecell}
\usepackage{xcolor}


%UL set section header spacing
\usepackage{titlesec}
% 
\titlespacing\subsubsection{0pt}{24pt plus 4pt minus 2pt}{0pt plus 2pt minus 2pt}


%UL set whitespace around verbatim environments
\usepackage{etoolbox}
\makeatletter
\preto{\@verbatim}{\topsep=0pt \partopsep=0pt }
\makeatother


%%%%% SELECT YOUR DRAFT OPTIONS
% This adds a "DRAFT" footer to every normal page.  (The first page of each chapter is not a "normal" page.)

% IP feb 2021: option to include line numbers in PDF

% for line wrapping in code blocks
\usepackage{fancyvrb}
\usepackage{fvextra}
\DefineVerbatimEnvironment{Highlighting}{Verbatim}{breaklines=true, breakanywhere=true, commandchars=\\\{\}}

% This highlights (in blue) corrections marked with (for words) \mccorrect{blah} or (for whole
% paragraphs) \begin{mccorrection} . . . \end{mccorrection}.  This can be useful for sending a PDF of
% your corrected thesis to your examiners for review.  Turn it off, and the blue disappears.
\correctionstrue


%%%%% BIBLIOGRAPHY SETUP
% Note that your bibliography will require some tweaking depending on your department, preferred format, etc.
% If you've not used LaTeX before, I recommend reading a little about biblatex/biber and getting started with it.
% If you're already a LaTeX pro and are used to natbib or something, modify as necessary.
% Either way, you'll have to choose and configure an appropriate bibliography format...

% this enables pandoc citations
\newlength{\cslhangindent}
\setlength{\cslhangindent}{1.5em}
\newlength{\csllabelwidth}
\setlength{\csllabelwidth}{3em}
\newlength{\cslentryspacingunit} % times entry-spacing
\setlength{\cslentryspacingunit}{\parskip}
\newenvironment{CSLReferences}[2] % #1 hanging-ident, #2 entry spacing
 {% don't indent paragraphs
  \setlength{\parindent}{0pt}
  % turn on hanging indent if param 1 is 1
  \ifodd #1
  \let\oldpar\par
  \def\par{\hangindent=\cslhangindent\oldpar}
  \fi
 }%
 {}
\usepackage{calc}
\newcommand{\CSLBlock}[1]{#1\hfill\break}
\newcommand{\CSLLeftMargin}[1]{\parbox[t]{\csllabelwidth}{#1}}
\newcommand{\CSLRightInline}[1]{\parbox[t]{\linewidth - \csllabelwidth}{#1}\break}
\newcommand{\CSLIndent}[1]{\hspace{\cslhangindent}#1}

\usepackage[style=authoryear, sorting=nyt, backend=biber, maxcitenames=2, useprefix, doi=true, isbn=false, uniquename=false]{biblatex}

\addbibresource{bibliography/references.bib}
\addbibresource{bibliography/additional-references.bib}

% This makes the bibliography left-aligned (not 'justified') and slightly smaller font.
\renewcommand*{\bibfont}{\raggedright\small}




% Uncomment this if you want equation numbers per section (2.3.12), instead of per chapter (2.18):
%\numberwithin{equation}{subsection}


%%%%% THESIS / TITLE PAGE INFORMATION
% Everybody needs to complete the following:
\renewcommand\title{Thesis Title}
\renewcommand\author{Author Name}
\newcommand\school{Your School}
\newcommand\orcid{0000-0000-0000-0000}%
\newcommand{\university}{Curtin University}
\newcommand{\submittedtext}{This thesis is presented for the Degree of}
\newcommand\degree{Doctor of Philosophy}
\newcommand\submissiondate{\today}


%%%%% YOUR OWN PERSONAL MACROS
% This is a good place to dump your own LaTeX macros as they come up.

% To make text superscripts shortcuts
	\renewcommand{\th}{\textsuperscript{th}} % ex: I won 4\th place
	\newcommand{\nd}{\textsuperscript{nd}}
	\renewcommand{\st}{\textsuperscript{st}}
	\newcommand{\rd}{\textsuperscript{rd}}

%%%%% THE ACTUAL DOCUMENT STARTS HERE
\begin{document}

%%%%% CHOOSE YOUR SECTION NUMBERING DEPTH HERE
% You have two choices.  First, how far down are sections numbered?  (Below that, they're named but
% don't get numbers.)  Second, what level of section appears in the table of contents?  These don't have
% to match: you can have numbered sections that don't show up in the ToC, or unnumbered sections that
% do.  Throughout, 0 = chapter; 1 = section; 2 = subsection; 3 = subsubsection, 4 = paragraph...

% The level that gets a number:
\setcounter{secnumdepth}{2}
% The level that shows up in the ToC:
\setcounter{tocdepth}{1}

% JEM: Pages are roman numbered from here, though page numbers are invisible until ToC.  This is in
% keeping with most typesetting conventions.
\begin{romanpages}

% Title page is created here
\maketitle

%%%%% DEDICATION

  \addcontentsline{toc}{chapter}{Frontmatter}

%%%%% DECLARATION
      \addcontentsline{toc}{section}{Declaration}
    \chapter*{Declaration}
  \noindent\emph{To the best of my knowledge and belief this thesis contains no
material previously published by any other person except where
due acknowledgement has been made.} \\

\noindent\emph{This thesis contains no material
that has been accepted for the award of any other degree or diploma
in any university.}\\

\vspace{3cm} 
\hspace{80mm}\rule{40mm}{.15mm}\par
\makeatletter
\hspace{80mm} \author \par
\hspace{80mm} \submissiondate
\makeatother

%%%%% ABSTRACT
    \chapter*{Abstract}
  Abstract text goes here - edit the file frontmatter/acknowledgements.tex

%%%%% SECOND ABSTRACT
    \chapter*{Resumen}
  Second abstract text goes here (e.g. to have an abstract in a second language) - edit the file frontmatter/second-abstract.tex

Disable the second abstract by commenting out the relevant lines in the FRONTMATTER section of index.Rmd

%%%%% ACKNOWLEDGEMENTS
      \addcontentsline{toc}{section}{Acknowledgements}
    \chapter*{Acknowledgements}
  Acknowledgements text goes here - edit the file frontmatter/acknowledgements.tex

%%%%% ACKNOWLEDGEMENT OF COUNTRY
      \addcontentsline{toc}{section}{Acknowledgement of Country}
    \chapter*{Acknowledgement of Country}
  We acknowledge that Curtin University works across hundreds of traditional lands and custodial groups in Australia, and with First Nations people around the globe. We wish to pay our deepest respects to their ancestors and members of their communities, past, present, and to their emerging leaders. Our passion and commitment to work with all Australians and peoples from across the world, including our First Nations peoples are at the core of the work we do, reflective of our institutions' values and commitment to our role as leaders in the Reconciliation space in Australia.

%%%%% LIST OF PUBLICATIONS
      \addcontentsline{toc}{section}{List of Publications}
    \chapter*{List of Publications}
  List of publications goes here - edit the file frontmatter/list-of-publications.tex

%%%%% LIST OF CONFERENCES
      \addcontentsline{toc}{section}{Conference Presentations}
    \chapter*{Conference Presentations}
  List of conferences goes here - edit the file frontmatter/list-of-conferences.tex

%%%%% STATEMENT OF CONTRIBUTIONS
      \addcontentsline{toc}{section}{Statement of Contribution of Others}
    \chapter*{Statement of Contribution of Others}
  Contribution of others text goes here - edit the file frontmatter/statement-of-contributions.tex

\vspace{1cm}
\begin{raggedleft} 
  Signature of candidate:  \hspace{0.2cm}\rule{45mm}{.15mm} \\[1cm]
  Signature of supervisor: \hspace{0.2cm}\rule{45mm}{.15mm} \\ 
\end{raggedleft}



%%%%% MINI TABLES
% This lays the groundwork for per-chapter, mini tables of contents.  Comment the following line
% (and remove \minitoc from the chapter files) if you don't want this.  Un-comment either of the
% next two lines if you want a per-chapter list of figures or tables.

% This aligns the bottom of the text of each page.  It generally makes things look better.
\flushbottom

% This is where the whole-document ToC appears:
\addcontentsline{toc}{section}{Contents}
\tableofcontents

\addcontentsline{toc}{section}{List of Figures}
\makeatletter
\@starttoc{lof}
\makeatother
% \mtcaddchapter is needed when adding a non-chapter (but chapter-like) entity to avoid confusing minitoc

% Uncomment to generate a list of tables:
\addcontentsline{toc}{section}{List of Tables}
\makeatletter
\@starttoc{lot}
\makeatother

%%%%% LIST OF ABBREVIATIONS
      \addcontentsline{toc}{section}{List of Abbreviations}
    \chapter*{List of Abbreviations}
  \begin{description}
\item[1-D, 2-D] One- or two-dimensional, referring **in this thesis** to spatial dimensions in an image.
\item[Otter] One of the finest of water mammals.
\item[Hedgehog] Quite a nice prickly friend.
\end{description}


% The Roman pages, like the Roman Empire, must come to its inevitable close.
\end{romanpages}

%%%%% CHAPTERS
% Add or remove any chapters you'd like here, by file name (excluding '.tex'):
\flushbottom

% all your chapters and appendices will appear here
\hypertarget{introduction}{%
\chapter{Introduction}\label{introduction}}

\adjustmtc
\markboth{Introduction}{}

Welcome to \texttt{oxforddown} (\protect\hyperlink{ref-lyngsOxforddown2019}{Lyngs, 2019}), a thesis template for R Markdown that I created when writing \href{https://ulyngs.github.io/phd-thesis/}{my own PhD thesis} at the University of Oxford.
This template allows you to write in R Markdown, while formatting the PDF output with the beautiful and time-tested \href{https://github.com/mcmanigle/OxThesis}{OxThesis LaTeX template}.
The sample content is partly adapted from \href{https://github.com/ismayc/thesisdown}{\texttt{thesisdown}} .

Hopefully, writing your thesis in R Markdown will provide a nicer interface to the OxThesis template if you haven't used TeX or LaTeX before.
More importantly, \emph{R Markdown} allows you to embed chunks of code within your thesis and generate plots and tables directly from the underlying data, avoiding copy-paste steps.
This gets you into the habit of doing reproducible research, which will benefit you long-term as a researcher, and also help anyone that is trying to reproduce or build upon your results down the road.

\hypertarget{why-use-it}{%
\section*{Why use it?}\label{why-use-it}}
\addcontentsline{toc}{section}{Why use it?}

\emph{R Markdown} creates a simple and straightforward way to interface with the beauty of LaTeX.
Packages have been written in \textbf{R} to work directly with LaTeX to produce nicely formatting tables and paragraphs.
In addition to creating a user friendly interface to LaTeX, \emph{R Markdown} allows you to read in your data, analyze it and to visualize it using \textbf{R}, \textbf{Python} or other languages, and provide documentation and commentary on the results of your project.\\
Further, it allows for results of code output to be passed inline to the commentary of your results.
You'll see more on this later, focusing on \textbf{R}.
If you are more into \textbf{Python} or something else, you can still use \emph{R Markdown} - see \href{https://bookdown.org/yihui/rmarkdown/language-engines.html}{`Other language engines'} in Yihui Xie's \href{https://bookdown.org/yihui/rmarkdown/language-engines.html}{\emph{R Markdown: The Definitive Guide}}.

Using LaTeX together with \emph{Markdown} is more consistent than the output of a word processor, much less prone to corruption or crashing, and the resulting file is smaller than a Word file.
While you may never have had problems using Word in the past, your thesis is likely going to be about twice as large and complex as anything you've written before, taxing Word's capabilities.

\hypertarget{who-should-use-it}{%
\section*{Who should use it?}\label{who-should-use-it}}
\addcontentsline{toc}{section}{Who should use it?}

Anyone who needs to use data analysis, math, tables, a lot of figures, complex cross-references, or who just cares about reproducibility in research can benefit from using \emph{R Markdown}.
If you are working in `softer' fields, the user-friendly nature of the \emph{Markdown} syntax and its ability to keep track of and easily include figures, automatically generate a table of contents, index, references, table of figures, etc. should still make it of great benefit to your thesis project.

\startappendices

\hypertarget{the-first-appendix}{%
\chapter{The First Appendix}\label{the-first-appendix}}

This first appendix includes an R chunk that was hidden in the document (using \texttt{echo\ =\ FALSE}) to help with readibility:

\textbf{In 02-rmd-basics-code.Rmd}

\textbf{And here's another one from the same chapter, i.e.~Chapter \ref{code}:}

\hypertarget{the-second-appendix-for-fun}{%
\chapter{The Second Appendix, for Fun}\label{the-second-appendix-for-fun}}

\hypertarget{refs}{}
\begin{CSLReferences}{1}{0}
\leavevmode\vadjust pre{\hypertarget{ref-lyngsOxforddown2019}{}}%
Lyngs, U. (2019). Oxforddown: An oxford university thesis template for r markdown. In \emph{GitHub repository}. \url{https://github.com/ulyngs/oxforddown}; GitHub. \url{https://doi.org/10.5281/zenodo.3484682}

\end{CSLReferences}

%%%%% REFERENCES
{\renewcommand*\MakeUppercase[1]{#1}%
\addcontentsline{toc}{chapter}{Bibliography}
\chapter*{Bibliography}
\printbibliography[heading=none]}


\end{document}
